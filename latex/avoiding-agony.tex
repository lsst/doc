
\newcommand{\code}[1]{\texttt{#1}}
\newcommand{\scons}{\code{scons}}
\newcommand{\make}{\code{make}}

\newcommand{\heading}[1]{\paragraph{#1}\rule{0pt}{0pt}}

\newenvironment{codesize}
               {\begin{small}}
               {\end{small}}
%               {\begin{scriptsize}}
%               {\end{scriptsize}}

\newenvironment{transcript}%
               {%\par\noindent\ignorespaces%
                 \vspace{4pt}\par\noindent%
                 \rule{4pt}{4pt}\hspace{1cm}%
                 \begin{minipage}[t]{0.8\textwidth}%
                   \begin{codesize}%
                     %\begin{verbatim}}%
                     \tt}%
               %{\end{verbatim}
               {%
                   \end{codesize}%
                 \end{minipage}%
                 \noindent\ignorespacesafterend\vspace{8pt}\par}

\section{How to avoid major sources of agony while developing LSST code}

\subsection{\scons\ likes to waste your time}

The \scons\ policy is to guarantee a correct build by default.  That
means doing a lot of work, which means wasting a lot of your time
checking to see whether you did an operating system upgrade since your
last recompile, 10 seconds ago.  Hey, it could happen.

\heading{Use \scons\ in interactive mode.}

\begin{transcript}
\begin{verbatim}
> scons --interact
scons: Reading SConscript files ...
Checking if  CC is really gcc...(cached) yes
[...]
scons: done reading SConscript files.
scons>>>
\end{verbatim}
\end{transcript}

\noindent
then build \emph{only the targets you need} using the \code{build} command:

\begin{transcript}
\begin{verbatim}
scons>>> build python
\end{verbatim}
\end{transcript}

\noindent or

\begin{transcript}
\begin{verbatim}
scons>>> build tests/testMyCode
\end{verbatim}
\end{transcript}




\heading{Use as many CPUs as you can get your hands on.}

\begin{transcript}
\begin{verbatim}
scons -j 8
\end{verbatim}
\end{transcript}

\noindent
Ignore all complaints that you're hogging the computer.  You're an
LSST developer.  You're worth it.





\heading{Avoid recomputing dependencies every time.}

\begin{transcript}
\begin{verbatim}
scons --implicit-cache --implicit-deps-unchanged
\end{verbatim}
\end{transcript}

\noindent
This tells \scons\ not to re-scan the source files looking for
\code{#include} statements in order to determine dependencies.  If you
actually do change the dependencies, you'll want to run \scons\ in
``vanilla'' mode to pick up those changes.


\subsection{\scons\ is a delicate princess}

Don't use \code{ctrl-C} to interrupt \scons\ during its configuration
phase.  It will cache incorrect configuration information.  In
practice, the only solution is then to do the excruciating:

\begin{transcript}
\begin{verbatim}
rm -Rf .scon*
\end{verbatim}
\end{transcript}

\noindent
in order to clear the caches.  This then means that \scons\ will
believe that everything needs to be recompiled.  Ouch.


