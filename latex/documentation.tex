% documentation

\renewcommand{\LaTeX}{LaTeX\xspace}

\newcommand{\tttex}{\texttt{.tex}\xspace}
\newcommand{\ttlsstpackages}{\texttt{lsstPackages.tex}\xspace}
\newcommand{\ttinput}{\texttt{\textbackslash input}\xspace}
\newcommand{\ttlsstmanual}{\texttt{lsstManual.tex}\xspace}
\newcommand{\ttdevenvdoc}{\texttt{devenv/doc}\xspace}
\newcommand{\tteups}{\texttt{eups}\xspace}
\newcommand{\ttsetup}{\texttt{setup}\xspace}
\newcommand{\ttscons}{\texttt{scons}\xspace}
\newcommand{\ttlsstinput}{\texttt{\textbackslash lsstinput}\xspace}
\newcommnad{\ttlatex2html}{\texttt{latex2html}\xspace}

\newcommand{\ttdoxygenconf}{\texttt{doxgyen.conf}\xspace}

% latex, html and doxygen
\section{Documentation Overview\label{sec-doc-overview}}

LSST software is documented in two fundamental ways:

\begin{description}
  
\item[The LSST Developer's Manual] This is a single document,
  written in \LaTeX, and is intended to be a `manual' in the traditional
  sense of the word.  It's available online in HTML format, and also
  as a PDF document. You're currently reading the Developer's Manual!

\item[The Doxygen source code documentation] This is a source code
  reference, containing summaries of the various classes and
  functions, and their associated parameters (ie. arguments, return
  values, etc.).  It is an HTML (online) reference which allows a
  developer to look up the application programming interface (API) for
  a given class or function, and to link quickly to other related
  classes and parameters.  Doxygen is machine generated directly from
  source code, with additional information (variable definitions,
  brief summaries, etc) taken from specific comments included by the
  programmer. Because of the way it is used (`browsing' source code),
  it is available only in HTML.  For more information on Doxygen, see
  \url{http://www.doxygen.org}.
    
\end{description}

Each of these forms of documentation are available for individual
packages, and for the entire LSST project.  Here, we'll describe the
details of the LSST documentation system.


% main doc package
\section{The Main Documentation Package, \ttdevenvdoc\label{sec-maindoc}}

The main documentation package lives in \ttdevenvdoc, and much
of the documentation contained in the Developer's Manual can be found
here.  However, the parts of the manual dealing with individual
packages are stored with the relevant packages, and are pulled into
the manual when it is compiled.

Conversely, there is very little Doxygen documentation stored in
\ttdevenvdoc.  When built, the package collects Doxygen
configuration files from all available (\ttsetup) packages and creates a
single comprehensive Doxygen reference.

The documentation package differs in structure from other standard
LSST packages, containing only \texttt{doxygen/} and \texttt{latex/}
directories.

\subsection{The Main \LaTeX Documentation\label{sec-mainlatex}}

The highlights of the \texttt{latex/} directory include:

\begin{description}
  \item [\ttlsstmanual] This is the central \LaTeX document that gets
    built.  However, it contains only \ttinput
    statements for the various chapters.
  \item [\ttlsstpackages] This is a machine-generated file which
    \ttinput s the \LaTeX{} documentation for the
    individual LSST packages.  Its creation is handled in the
    \texttt{latex/SConscript} file, where the absolution paths to the
    currently setup package documentation are prepended to the
    \ttinput filename.
  \item [{\itshape everything-else}\tttex] The remaining
    *\tttex files contain the chapters which are imported
    into \ttlsstmanual.
  \item[\texttt{figures/}] This subdirectory contains any figures
    used in the manual.
  \item[\texttt{htmlDir/}] This subdirectory contains an HTML
    version of the manual, generated with \ttlatex2html.  The
    directory and its contents will not exist before \ttscons is
    run.
\end{description}

The importing of the individual package documentation requires a bit
of explanation.  The system is designed such that the final document
which gets built (by scons) will pull in the individual package
documentation from any packages which were \ttsetup (using the
\tteups `\ttsetup' command).

As alluded to in the file explanations above, these \tttex files for
each package are imported in the \ttlsstpackages file.
During the build, the \texttt{latex/SConscript} code determines which
packages are \ttsetup (and {\itshape where}) and generates the
\ttlsstpackages file with an \ttinput statement for each.  However, if
you look in the \ttlsstpackages file, that's not what you'll see.

It's entirely possible that each imported \tttex file will also
contain \ttinput statements for files in its own directory tree.  To
deal with this, {\itshape all individual packages must only use our
  own \ttlsstinput command!}  The \ttlsstinput command is a thin
wrapper for the regular \LaTeX \ttinput command.  For each imported
package, the \ttlsstpackages file contains two lines: the first renews
the \ttlsstinput command to prepend the absolution path for the
package in question, the second then \ttlsstinput's the package.

When \ttscons is run, the \LaTeX is compiled and
\texttt{lsstManual.pdf} is created in the root directory of
\ttdevenvdoc.  At the same time, an HTML version of the manual will be
created in \texttt{latex/htmlDir}.

\subsection{The Main Doxygen Documentation\label{sec-maindoxygen}}

Doxygen builds its documentation by digesting the source code for a
project.  The construction of the LSST doxygen is controlled by a
\ttdoxygenconf file which identifies the locations (directories) of the
source code to be scanned.  The \ttdevenvdoc \texttt{doxygen/}
directory contains no \ttdoxygenconf file, but rather a python script
\texttt{writeDoxyConfig.py} which is executed when \ttscons is run.
As you might suspect, \texttt{writeDoxyConfig.py} scans all available
\ttdoxygenconf files (ie. those \ttsetup via \tteups) and generates a
master \ttdoxygenconf.  This machine-generated file is then used to
build the top-level Doxygen.  This happens when \ttscons is run and is
handled in the \texttt{doxygen/SConscript} file.

When \ttscons is run, the Doxygen output will be created in
\texttt{doxygen/htmlDir}.


% individual package documentation
\section{Documenting an Individual Package\label{sec-packagedoc}}



