% documentation

% latex, html and doxygen
\section{Documentation Overview\label{sec-doc-overview}}

LSST software is documented in two fundamental ways:

\begin{description}
  
\item[The LSST Developer's Manual] This is a single document,
  written in \latex, and is intended to be a `manual' in the traditional
  sense of the word.  It's available online in HTML format, and also
  as a PDF document. You're currently reading the Developer's Manual!

\item[The Doxygen source code documentation] This is a source code
  reference, containing summaries of the various classes and
  functions, and their associated parameters (ie. arguments, return
  values, etc.).  It is an HTML (online) reference which allows a
  developer to look up the application programming interface (API) for
  a given class or function, and to link quickly to other related
  classes and parameters.  Doxygen is machine generated directly from
  source code, with additional information (variable definitions,
  brief summaries, etc) taken from specific comments included by the
  programmer. Because of the way it is used (`browsing' source code),
  it is available only in HTML.
    
\end{list}

Each of these forms of documentation are available for individual
packages, and for the entire LSST project.  Here, we'll describe the
details of the LSST documentation system.


% main doc package
\section{The Main Documentation Package, \texttt{devenv/doc}\label{sec-maindoc}}

The main documentation package lives in \texttt{devenv/doc}, and much
of the documentation contained in the Developer's Manual can be found
here.  However, the parts of the manual dealing with individual
packages are stored with the relevant packages, and are pulled into
the manual when it is compiled.

Conversely, there is very little Doxygen documentation stored in
\texttt{devenv/doc}.  When built, the package collects Doxygen
configuration files from all available (setup) packages and creates a
single comprehensive Doxygen reference.

The documentation package differs in structure from other standard
LSST packages, containing only \texttt{doxygen/} and \texttt{latex/}
directories.

\subsection{The Main \latex Documentation}




% individual package documentation
\section{Documenting an Individual Package\label{sec-packagedoc}}


